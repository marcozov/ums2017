\documentclass[a4paper]{article}

\usepackage[english]{babel}
\usepackage[utf8]{inputenc}
\usepackage{amsmath}
\usepackage{graphicx}
\usepackage{verbatim}
\usepackage[colorinlistoftodos]{todonotes}
\title{Swiss Neutrality during the Cold War}

\author{Marco Zoveralli, Linsay Pinheiro, Artem Shevchenko, Luis Medina Rios}

\date{\today}

\begin{document}
\maketitle

\begin{abstract}Swiss neutrality is often regarded as one of the most important characteristics of Switzerland. Although it originated as something very vague, quite superficial, and that depended more on surround powers rather than on  Switzerland itself, there are several examples of how this “Swiss pillar”, which began gaining importance for the sake of the Swiss image during the two world wars, positively affected the country under many different aspects as well. 
\\*In this report, we focus on the role of Switzerland in the Cold War period, by putting an accent on the impact of Swiss neutrality during some of the major events that involved this nation. We argue that during this period the position of Switzerland was controversial and that its neutrality was in fact biased, due to its tendency to favor the Western Bloc in more than one occasion. 
We describe different events in which Switzerland, besides its ambiguous position, managed to get relevant benefits from several perspectives. First, we focus on showing how Swiss neutrality is highly arguable in the context of the arms control policy in relation to the two blocs. In particular, we discuss both export and import arms policies and we put the accent on Swiss neutrality, by pointing out how questionable it was and how this "pillar" was actually an instrument to get benefits and the sympathy of the Western Bloc. 
Then, we discuss the recognition of Asian countries, by pointing out how "universality" was violated, since North Korea and North Vietnam were not recognized. Again, we show how this was an additional effort to better integrate with the Western Bloc.
Finally, we describe the representation of the US interests in Cuba, while showing how this could still be considered as an attempt to get close to the Western Bloc.
\end{abstract}

\clearpage

\tableofcontents

\clearpage

\section{Introduction}
\label{sec:introduction}
In 1815, when Napoleon was defeated in the battle of Waterloo, Switzerland was recognized as neutral by France and Austria. Besides this vague origin of neutrality, this characteristic gradually gained importance in the following century, when Switzerland and the surrounding powers benefited from the Swiss impartial position. In fact, especially at the beginning of it, neutrality depended more on these surrounding countries, rather than Switzerland itself. This trend did not really change until the beginning of 20th century: although Swiss neutrality was mentioned in few additional occasions – namely in the constitution, when Switzerland became a federal state, and in the Peace Conferences in The Hague –, the country itself has never been very serious about that.
\\*\\*World War I was a turning point: in this occasion, neutrality became a pillar of the Swiss identity. One enormous benefit that Switzerland gained was from the economic perspective: this is when the country rose as a financial center. Some reasons for this growth were the stability and the strengthening of the Swiss franc, along with the bank secrecy. These characteristics attracted several wealthy people from foreign countries to deposit their money in Swiss banks: this is why banks became so big.
\\*\\*During World War II, Switzerland declared itself neutral again. Differently from other countries that self-declared neutrality, such as Belgium, Switzerland was not invaded. Traditional explanations for this phenomenon regard to the fact that Switzerland was neutral and that Swiss army was prepared to contrast any kind of invasion. Deeper analyses showed that the both the Axis and Allied powers got benefits from Switzerland. For instance, they were both importing weapons from it. Also, Switzerland kept playing the role of financial center. Furthermore, it played a key role as a transportation axis between Germany and Italy.
\\*\\*Although Switzerland did not participate in the World War II, it was in a very bad position in the subsequent years, because of its support to the Axis power. It was criticized by the allied power and by the USA. For this reason, Switzerland put much effort in order to improve the image that was ruined due to the previous years' events: it increased its efforts to show its solidarity during the Cold War, like its contribution to the Marshall Plan. Besides the critics that have been done to Switzerland, the country managed to improve its reputation. In fact, neutrality was not only kept, but it also provided noticeable profit. 
\\*\\*In this report we analyze the position of Switzerland during some important phases of the Cold War. We show how ambiguous the position of this country was in some circumstances, what actions allowed Switzerland to keep its reputation at an acceptable level, and the benefits that it managed to obtain. In particular, we focus on three distinct and relevant events.
\\*\\*First, we examine how Switzerland changed its politics during the Cold War in relation to the arms industry, by pointing out some important aspects about the consequences on Swiss military/economic interests and neutrality. We show the behavior of Switzerland in terms of arms import and exports, by discussing the sustainability of neutrality in relation to the way in which it interacted with the two Blocs. We will argue that the position of Switzerland aimed to get the sympathy of the USA and that it resulted to be against the URSS.
\\*\\*Then, we discuss the arguable neutral position in the context of the recognition of Asian countries. The one of Petitpierre's maxims was the maxim of universality, according to which any state was supposed to be recognized despite its political regime. However, the situation during Cold War influenced greatly the realization of this maxim (in fact, it is better to refer to it as a pragmatic universality). For economical (economical ties with countries in the Western Bloc) and political (Switzerland was western-oriented neutral and tried to maintain good relation with the USA and the Western Bloc) reasons, Switzerland recognized only non-communist parts of many divided states for a long time. We will describe the very slow process of recognition of North Korea and North Vietnam which took many years to illustrate it.
\\*\\*Finally, we explore the paradoxical neutrality that Switzerland had when it was representing US interests in Cuba during the Cold War, by explaining the main reasons on why the United States chose Switzerland and why they accepted to do it, showing a very clear Swiss inclination towards the Western Bloc. We also discuss the crucial role that Switzerland played as a protective power and the US interests that they were protecting as well as the consequences that Switzerland had due to its relation with the US.
\clearpage
\section{Swiss Arms Control Policy and Neutrality}
During the two worlds wars, Switzerland managed to keep its neutrality without being invaded, differently from other countries. For instance, Belgium was invaded during both wars, besides its non-participation at the beginning of both wars [\ref{reference1}][\ref{reference2}]. This was mainly accomplished because the surrounding powers had benefits from Switzerland being neutral. As already mentioned, the Axis powers gained several benefits during both the world wars [\ref{reference3}]. This is also a noticeable phase in which neutrality was clearly violated [\ref{reference3}]. In the context of arms control, Switzerland had a strange position for more than a reason. First, being neutral might imply, from a certain perspective, to limit the export of goods that may cause excessive damange to the other countries (especially if initiatives of this kind are taken from other powers). Second, although enforcing the Swiss military strenght makes sense --prevention is better than cure--, neutrality implies treating all the other countries exactly in the same way, especially when arms trade is involved. These two points will be discussed in the following paragraphs.
\subsection{Swiss Arms Control Policy: From Abstention to Participation}
One relevant matter that characterized the position of Switzerland was its relationship with the arms control policy and disarmament. While during the two world wars Switzerland did not take part in any of these activities, exception made for some initiatives within the framework of the League of Nations during the period between the wars [\ref{reference4}] --in fact the export rates of arms were huge, to both Allied and Axis powers [\ref{reference5}] --, at a certain point of the Cold War this phenomenon changed dramatically. The variation of the Swiss attitude towards the arms control was triggered in the 1960s by the advent of the Nuclear Non-Proliferation Treaty (NPT) [\ref{reference4}], whose goal is to achieve nuclear disarmament and to promote the peaceful uses of nuclear energy [\ref{reference6}]. The NPT caused several debates in Switzerland and it was signed only in 1969, with the condition that that the ratification would take place only when a significant number of countries had done the same [\ref{reference4}]. This happened after almost ten years more and Switzerland finally decided to officially support the treaty only in 1977 [\ref{reference4}]. Measures were taken also for what concerns the biological and chemical weapons, when an entire category of weapons was banned through the Biological and Toxin Weapons Convention in 1975 [\ref{reference7}].
\subsection{Limiting the Exports: the "Wait and See" Policy}
Besides the events described in the previous section, arms control was not really part of the Swiss security policy until the end of the Cold War [\ref{reference9}]. In fact, strictly speaking, Switzerland had no real interest in enhancing its participation in arms control negotiation. This is was often justified by pointing out the neutral position of Switzerland and its lack of interest in what concerns non-defensive actions [\ref{reference4}]. As a result, Switzerland ended up taking decisions only when it was forced by external factors: it focused more on keeping open as many options as possible, rather than getting an actual interest and comprehension of the importance of having a proper arms control policy and disarmament agreements [\ref{reference4}]. In particular, during the Cold War, this was dictated by the United States and by the Soviet Union, which were the two greatest powers.
\\*Due to its neutral position, during the two world wars, Switzerland handled the numerous arms exports individually with each state [\ref{reference4}]. This was enhanced during the Cold War, due to the pressure that came from the West Bloc. This is what made Switzerland sign the Hotz-Linder Agreemenet, whose goal was to restrict the exports of strategic goods to the East Bloc [\ref{reference10}]. In return, exports of strategic goods to Switzerland were allowed and, as we will show in [\ref{subsec:armsImport}], this had a significant impact on the questionable neutral position of Switzerland during the whole Cold War, due to its import policy. This can be also regarded as an infraction of the Swiss neutrality. \\*Another important event that involved Switzerland was the creation of the Nuclear Suppliers Good (NSG), which dates back to 1974 [\ref{reference11}] and whose goal was to control the exports policies of the main nuclear suppliers. Switzerland was interested in participating, since this would lead to: [\ref{reference4}] 
\begin{itemize}
\item{the possibility of seeing how the nuclear market trends developed}
\item{the possibility of participating in the definitive NSG guidelines}
\item{non-increasing pressure from the main nuclear suppliers}
\end{itemize}
In this section we showed how Switzerland's attitude was kind of opportunist in the context of arms export and how its position already tended to favor the West Bloc, and therefore its neutral position could be already argued. In the next section we describe how this attitude was even stronger in the context of arms import during the early Cold War, and we confirm that Swiss neutrality was violated in other ways, by giving more details on the matter.
\subsection{Arms import: The Relation with the West and East Blocs during the early Cold War}
\label{subsec:armsImport}
As already mentioned in the Introduction [\ref{sec:introduction}], after World War II, Switzerland was strongly criticized by the allied powers \footnote{It was also defined as a "war-profiteer" that contributed to the growth of the Third Reich}. Initially, after the war, the main powers intended to limit neutrality dramatically. The Charter of United Nations argued that neutrality looked more like a policy of war, rather than a policy of peace, and that the Charter's decisions had to be accepted, even though they would violate neutrality sometimes [\ref{reference12}]. Soviet Union was reluctant as well, mainly due to the Swiss anti-communist position.
Swiss neutrality was reconsidered by the two blocs at the beginning of the Cold War: the USA considered Switzerland as a strong and useful ally, and the Soviet Union implicitly recognized Swiss neutrality by signing Austrian State Treaty in 1955, in order to avoid a hypothetical situation in which Austria and Switzerland would both join the NATO. 
\\*During the first years of the Cold War, Switzerland heavily relied on leading countries of the West Bloc for what concerns the imports of arms [\ref{reference12}]. This dependency on imports of this kind leads to an obvious question: was neutrality feasible if Switzerland was depending so much on the West Bloc? As pointed out by Marco Wyss in [\ref{reference12}], there are discordant opinions on the matter. 
\\*On the one hand, it can be argued that dependency on weapons does not necessarily compromise neutrality, as long as they remain under the control of the neutral buyer. Furthermore, according to this point of view, weapons could be considered as generic goods that can be traded uniquely for economic purposes. 
\\*On the other hand, this view is clearly questionable in terms of "credibility and respectability" [\ref{reference14}], which are two of the main principles on which neutrality is strongly based. As a permanent neutral state, Switzerland committed itself not to start/join any war, unless it was invaded. However, permanent neutrality also implies that nothing should be done in order to make it questionable in the future [\ref{reference16}]. All of this is even more debatable if we consider that Swiss neutrality, besides what the national authorities may or might have stated, has always been a mean rather than a purpose [\ref{reference15}].
We agree with this second opinion.
\subsubsection{Switzerland's armed neutrality: a Paradox}
The "armed neutrality" refers to the fact that Switzerland aims to keep its military strenght to an acceptable level. While this can be a reasonable point of view, since being prepared to defend themselves would partially prevent possible invasions, the arguable point relates to how this armed neutrality was achieved. 
Initially, the Swiss attitude was to be able to be self-sufficient for what concerns the arms production: they were conscious of the fact that any kind of dependency with foreign countries would not lead to anything good. As a matter of fact, Switzerland did not manage to keep up with the technological developments in relation to both the aircraft and tanks industry.
\\*However, Switzerland had to rely on foreign countries to meet its needs in terms of amount of weapons [\ref{reference19}]. Any neutral country would equally purchase the required weapons from any country that is able to provide them. This was not the case of Switzerland. In fact, almost all of the tanks and aircrafts were bought from the United Kingdom [\ref{reference13}], with few exceptions that included France. %page 5 on marco wyss article (neutrality in the early cold war): it says interesting things on whether this could be considered as a violation of neutrality
This was an important factor that contributed in strengthening the relationship the UK and Switzerland, which was already consistent from a political and economical perspective [\ref{reference12}]. This integration into the western arms transfer system was not casual: Switzerland meticulously tried to get into it more than once and it eventually succeed in 1945, when the allied powers were attracted by the advantages that Switzerland could bring into their economy. Yet, for several years, the USA refused to accept Switzerland: it judged it as a "war-profiteer" and it did not like the idea of neutrality. Strictly speaking, just after World War II, Switzerland was not considered so much differenly from countries such as Germany and the future Soviet satellites, by the State-War-Navy Coordinating Committee [\ref{reference12}], which issued a paper that contained the following text: "it is not considered consistent with United States policy to support with United States military supplies the armed forces of Poland, Romania, Bulgaria, Yugoslavia, Austria, Hungary, Albania, Spain, Finland, \textbf{Switzerland}, and, of course, Germany". 
Things began changing gradually as the tension among West and the East kept rising. At the starting phase of this change, in 1947, America accepted to release some categories of aircrafts to Switzerland: the State Department accepted the Swiss request that related to the provision of 100 P-51s. This was a first sign of improvement for the US-Swiss relations --probably due to the recognition of the fact that Switzerland was standing on the West Block's side-- and Switzerland also decided to participate in the Marshall Plan [\ref{reference19}]. % should be reference93 of Marco wyss article
Yet, Switzerland was still considered a customer to serve with "no priority": the Swiss access to the American military resources was limited to the aircraft. This trend did not change much in the following years: several Swiss requests were refused from the USA and therefore Switzerland kept relying on the UK and France at the end of the 1940s and during the early 1950s. One relevant example relates to the Swiss request for several tanks in 1949. Although the State Department, the NATO Supreme Commander, and the US President Dwight Eisenhower argued in favor of the Swiss request, the Department of Defense rejected it, because it feared of an interference with the production of tanks [\ref{reference19}]. % need reference
The priority status related to Switzerland was still the same in 1951, since it was expected to remain neutral if a war occurred [\ref{reference20}]. Yet, the Swiss Ministed in Washington was trying to convince Americans that being in good relation with Switzerland was a strategic asset, and that accepting to provide tanks to Switzerland would lead to gain advantages for both of them. The State Department agreed on this and it initiated a National Security Council (NSC) paper on Switzerland. Several points of this paper pointed out the advantages of mantaining a good relationship with Switzerland. First, its geographic position was considered of strategic importance for what concerns the security of Western Europe: it clearly represented an interest for the USA as well. Second, it was already obvious to the USA that Switzerland was considering the URSS as the only potential enemy and that the Swiss concept of neutrality was getting more and more flexible. As a corollary, Swiss neutrality tended to favour the West Bloc and a clear expectation from the USA was military aid, in case an attack from the URSS would occur [\ref{reference20}]. Finally, Swiss Army was among the strongest in Europe, but, as already pointed out, it was lacking of military hardware. This is why the paper recommended to increase the arms provision to Switzerland. Despite some disagreement --such as the one from the Joint Strategic Survey Committee--, Switzerland was considered eligible for military assistance. Yet, access to modern US weapons was granted only after two or three additional years. One of the reasons for this was that Switzerland did not sign any bilateral agreement with Washington, which was necessary in order to receive military assistance. This behavior was motivated by the fact that no formal agreement of this kind was really acceptable for a neutral country[\ref{reference20}]. Switzerland was fully integrated in the US system in the following years --around 1952--, when the USA kept becoming more and more interested in enforcing the security through Switzerland, as pointed out in the NSC.
\\*While this behavior of Switzerland clearly shows its intentions about the position to take among the two Blocs, it should also be noticed that this was enforced by the fact that imports from the East Bloc were repeteadly refused in more than one occasion. In fact, the trade with Soviet Union during the Cold War remained below one percent of the foreign trade and in general, Switzerland preferred to "react" to upcoming Soviet wishes, rather than playing an active role to improve their relation[\ref{reference18}].
%\\*On the other hand, imports from the East Bloc were repeteadly refused in more than one occasion. ... some text to explain this and to show that Switzerland was actually against the East.
%\subsubsection{Credibility: the Armed Neutrality Paradox}
%The credibility of Swiss neutrality was extremely undermined by their dependency on the USA. This is a fact. 
%\subsection{Neutrality: Was it Real and Coherent?}
\section{Switzerland and recognition of Asian countries}
\subsection{Petitpierre's universality}
In order to relegitimise Swiss neutrality compromised during WWII, Max Petitpierre, Switzerland’s Foreign Minister, formulated the new doctrine which consisted of three maxims: “Neutrality, solidarity and universality”. The idea behind the latter maxim was that Switzerland was supposed to recognize all states despite their political regime as long as the conditions -- a permanent population, a defined territory, a government and the capacity to enter into relations with other states -- were met []. However, Petitpierre himself noticed that this would not be an easy undertaking since the country’s destiny was tied to that of Europe []. 
\subsection{Reasons for the restriction of universality}
As the example of the real application of the 'universality' maxim, the establishment of official relations with the USSR can be considered. In this case Switzerland was interested in this relations as Petitpierre wanted to protect Swiss interests in the Eastern Europe and was interested in diplomatic relations with all major players in world politics.

In the Cold War Switzerland was a Western-oriented neutral, and integrated the Western Bloc an economic, political and military level []. This integration caused significant restrictions of universality application. Appearance of divided states created problem that both governments claimed to be the only rightful government of the divided state and the establishing relations with both sides as independent states was impossible without risking the future relationship with both or one of the regimes [].

In case of two Germanies – the Federal Republic of Germany (FRG) and the German Democratic Republic (GDR) – the West one was more important to Switzerland as those two countries had strong political and financial ties. Moreover, Bonn was following a policy (Hallstein doctrine) according to which the FRG would refuse to establish or pursue diplomatic relations with any government that recognised East Germany []. Because of this reason Switzerland recognised GDR only in 1972 whereas established official relations with FRG in 1951. One more example, which is unusual as far as recognition of communist part was preferred over the relations with non-communist part, was the recognition of the People’s Republic of China (PRC) (and, therefore, breaking relations with the Republic of China). The reason for such step was that Bern wanted to protect Switzerland’s economic interests in China. Also, in this particular case, the country was not divided into two halves as far as the PRC controlled the main part of Chinese territory.

In case of the countries of the Third World, Switzerland took up official relations with the southern non-communist 'halves' of Korea and Vietnam more than a decade before it recognised their communist parts. Despite economical interest there were several more reasons. First of all, Bern had the worry that the establishment of official relations with North Vietnam or North Korea could raise the question of the recognition of East Germany which would endanger important relations with the FRG. Secondly, there were the fear that an early recognition of North Vietnam and North Korea would create problems in relations with the United States. The Foreign Affairs Department explained in a strategy paper of 5 October 1964 that Switzerland adopted the position of recognising only one side of the divided states after 'consideration of Swiss interests' instead of following the 'universality' principle to the point: 'In Vietnam and Korea economic considerations very clearly speak in favour of relations with the Southern halves.'[]. One more important factor which influenced the recognition of communist parts – the anti-communist position of Swiss political authorities during the Cold War. This factor can be illustrated by a Foreign Affairs Department study of January 1962, which stated that 'we are not interested in having any further communist spying outposts in Berne', the one already existing was, according to this study, the PRC [].

So, in practice, the application of the 'universality' maxim was very pragmatic. If the establishment of diplomatic relations with a Communist state could lead to problems in the relationship with an important partner from Western bloc, the principle of universality was not applied. In the following parts of article the 'pragmatic' approach to the recognition of North Korea and North Vietnam will be described in details.
\subsection{Relations with North Vietnam}
Switzerland, for pragmatic reasons, recognised only the South part of Vietnam in 1958. The economical importance of ties with this non-communist part can be illustrated by figures: Switzerland traded US\$19.5 million worth of goods with South Vietnam, and only US\$1.3 million with North Vietnam. []

The things changed in 1966 with the Vietnam War. The Swiss government wanted to play a role in the diplomatic resolution of the conflict in order to increase the credibility of neutrality and began its diplomatic campaign. In order to compete with Sweden for the role of peace mediator in Vietnam, Switzerland had to improve relations with North Vietnam but, on the other hand, they did not want to take the step of official recognition. They tried to create political ties without complete diplomatic relations: the Swiss embassies in some countries approached the representatives of North Vietnam. However, Switzerland lost to France because they had not recognised North Vietnam and it was the important factor in the North Vietnam’s opposition to holding the talks in Geneva. After that Swiss policy-makers began to think about diplomatic relations with both Vietnams.

Sweden was the first Western European country which recognized North Vietnam on 10 January 1969. This step put the government of Switzerland under pressure and forced them to take action regarding North Vietnam. In the Swiss Parliament, a Social Democratic deputy questioned the government on the non-recognition of North Vietnam and argued that it is a contradiction with the maxim of universality []. International criticism of the USA policy in Vietnam had grown, and in Switzerland political youth movements attacked the Swiss government's association with Washington in the context of the Vietnam War. 

The Swiss Ambassador at Washington was sure that recognition will not create tensions in the relations with Washington if it was presented as a humanitarian step and would be beneficial for the international image of Swiss neutrality.

A first draft of the official recognition of North Vietnam was prepared by the foreign-affairs department in December 1970 but it was decided not to put it into action immediately because Saigon wished not to see this recognition at least until elections in South Vietnam in autumn 1971. However, it was the reputation of Swiss neutrality that was the most important factor: it would be necessary to be among the first Western countries who officially recognized Hanoi. That would speak well for the Swiss policy of universality and would allow Switzerland to participate in the country's reconstruction. Therefore, Bern eventually recognized North Vietnam on 1 September 1971.

So, the pragmatic approach in case of relations with North Vietnam is evident. The steps towards relations with this country are done only when not taking those steps can endanger the credibility of Swiss neutrality. Moreover, those steps are very careful to avoid possible conflict with the important partners from the Western bloc (like the USA and South Vietnam).
\subsection{Relations with North Korea}
As in case of Vietnam, Switzerland, for pragmatic reasons, recognised only the South non-communist part of Korea with which the official relations were established in 1962. During the 1960s, Swiss trade with South Korea totalled US\$38.4 million, while that with North Korea amounted to a mere US\$4.9 million.[]
The North Korea started to establish economic and diplomatic links to the non-communist states in the second half of 1960 and commercial relations with Pyongyang started to increase. Pyongyang proposed Switzerland to conclude a bilateral commercial and cultural agreement. In the beginning, the Swiss decided not to tighten its relations with North Korea but were stimulated by the Swedish and Austrian relations with Pyongyang and started to show the interest in economic ties.

Switzerland also played the role of mediator, for instance, during the ‘Pueblo’ incident when the capturing by North Korea of a US surveillance vessel took place and participated in Neutral Nations Supervisory Commission representing South Korea.

Sweden officially recognised North Korea on 6 April 1973, the same was done by Finland and Denmark. The Swiss politics closely followed the reactions of the Western powers to the Swedish step. However, Washington's reaction was not so angry as it was predicted. The Swiss Ambassador at Washington informed: 'The Swedish decision is received here with equanimity even if it is regretted. The State Department does not, however, seem to attach too much importance to it.' []

As the Swiss Ambassador to China said: 'The [North] Koreans have a bad character, they are not likable, they have a suffocating regime. But they exist, their economy is booming, they are solid.' [] The relations between countries continued to develop. Bern and North Korea agreed in July 1973 that Pyongyang would set up a commercial mission in Switzerland. The mission was opened in Zurich and not in Bern in order to avoid contradiction with South Korea and not to give this decision an official character. It had brought new business relations: North Korea was interested in acquiring several Swiss industrial plants. In June 1973, the South Korean President claimed that South Korea would not be against recognition of the two Koreas by any state. Finally, Switzerland recognised North Korea in December 1974.

To conclude, those facts perfectly illustrate the pragmatic approach in relations with North Korea. The great influence of economic relations on diplomatic ones is clear. And again, all steps toward relations were done carefully to avoid confrontation with partners from the Western bloc.
\clearpage
\section{Representation of US interests in Cuba}
In 3rd January 1961, the US government cut diplomatic links with Cuba after two years marked with worsening relationships between the United States and Cuba since Fidel Castro took the power. 
From that date, Switzerland has represented US interests in Havana, until 2015, when the United States and Cuba fully restored their relations which led to the end of Switzerland’s mandate. 
\subsection{Why did the USA choose Switzerland to represent their interests in Cuba?}
The decision of the United Stated to choose Switzerland to fulfill such a responsibility was quite unexpected. Indeed, the first thought of US executives would be naturally to delegate the representation of their interests to the United Kingdom, as both countries share similarities. 
However, the chosen country ended up being Switzerland for two main reasons:
\begin{itemize}
\item Contrary to the other candidates such as Canada and the United Kingdom, Switzerland is a neutral country, thus Cuba was more likely to accept the United States being represented by it;
\item Paradoxically, although Switzerland was not strictly speaking integrated in the Western bloc, it shared its ideas and despised communism, which clearly positioned the country in the US camp. Moreover, this is shown by the fact that when Fidel Castro took the power, Switzerland decided on its own to reduce economic exchanges with Cuba. Thus, the United States ensured that their interests would be properly represented. 
\end{itemize}
Therefore, the reason why the United States chose Switzerland to handle the crucial task of representing their interests in Cuba is, to sum up, their “paradoxical” neutrality during the Cold War. Indeed, it is positioned as a neutral country while showing a clear inclination towards the Western bloc. 
As for Switzerland, being invested with this responsibility represented a substantial advantage. First, offering its services would enable them not to be completely isolated during the Cold War and compensate the fact that they did not join the United Nations (UN). But above all, the main motivation of Switzerland to represent US interests in Cuba was the visibility that it would give to the country in the USA and all over the world, and also the economic benefits that would result from that agreement as the Unites States were one of Switzerland’s main economic partners. 
\subsection{Role of Switzerland as a protective power in Cuba}
Pressure was significant on Switzerland as if it failed to correctly represent US interests in Cuba, it was likely to provoke the anger of Washington. 
Moreover, Swiss neutrality can be endangered by Switzerland taking part in that way in the Cold War, even though originally, it is precisely their neutrality that allows them to act as an intermediary between two countries of opposite camps during the Cold War. 
The representation of US interests included the following missions, among others:
\begin{itemize}
\item protecting US citizens living in Cuba by offering financial assistance if needed for example;
\item visiting US prisoners;
\item making payments to the payees of federal benefits and perform notarial services;
\item forward visa and passport applications to Washington.
\end{itemize}
The Swiss delegation performed well the above-mentioned classic tasks but it is especially the “special” tasks, or crisis between the USA and Cuba that showed Switzerland’s crucial role as the protective power of US interests in Cuba. 
Details on how the Swiss delegation handled the following conflicts (and even sometimes the efforts they put to solve them were more than what the US government expected from them): 
\begin{itemize}
\item Between 1963 and 1964: Cuban government tried to nationalize the building that was previously sheltering the US embassy.
Efforts of the Swiss ambassador Stadelhofer to solve the issue without conflicts, invoking article 45 of the Vienna Convention on diplomatic relations. 
\item The Camarioca Crisis and the Varadero-Miami airlift 
Exceptional implication during eight years to transfer thousands of Cubans opposed to Castro regime to Florida (around 200 000): Main task of the mandate.
Long negotiations phase as the United Stated and Cuba could not come to an agreement. 
\item In May 1970, undergoing a three-day siege of US embassy’s old headquarters.
\end{itemize}
\subsection{Results on the relations between the United States and Switzerland}
Deepening of the relations between the two countries following the conflicts:
\begin{itemize}
\item Recognition of the political utility of "neutrality":  "If neutral Switzerland did not exist, we had to invent it" (Mc George Bundy, March 1962)
\item Change of position of the USA towards Switzerland’s integration in the European Economic Community (EEC). Originally, US government was opposed to Switzerland joining the EEC, stating that as Switzerland only wants to join for economic purposes, it would destroy the political content of the organization. However, US position changed progressively into a more positive one after considering the crucial role that Switzerland played in Cuba. 
\item Improvement of economic relationships: for instance, decrease of the taxes on watchmaking industry products. 
\end{itemize}
\section{Conclusion}

\addcontentsline{toc}{section}{References}
\begin{thebibliography}{9}
%\bibitem{nano3}
%  K. Grove-Rasmussen og Jesper Nygard,
%  \emph{Kvantefanomener i Nanosystemer}.
%  Niels Bohr Institute \& Nano-Science Center, Københavns Universitet

%\bibitem{nano3}
%  Aanchal Malhotra, Isaac E. Cohen, Erik Brakke, and Sharon Goldberg, \emph{Attacking the Network Time %Protocol}. Journal name, p. X-Y, October 2015.
\begin{comment}
\bibitem{nano3}
  Jurg Stussi-Lauterberg, \emph{Historical Outline on the question of Swiss Nuclear Armament}, December 1995
  
\bibitem{nano3}
  Christian B\"{u}hlmann, \emph{Le développement de l'arme atomique en Suisse}, June 2007

\bibitem{nano3}
  Switzerland during the Cold War (1945-1989). https://www.eda.admin.ch/aboutswitzerland/en/home/geschichte/epochen/die-schweiz-im-kalten-krieg--1945-1989-.html
  
\bibitem{nano3}
  Frédéric Joye-Cagnard, \emph{Des effects de la politique de la sciece Americaine Internationale sur la construction de la politique de la science en Suisse (1945-1960)}. Wissenschaft und Aussenpolitik, p. 35-46, 2012  

\bibitem{nano3}
Conference des Ministres, Max Petitpierre, Department of Foreign Affairs, http://dodis.ch/321, 12 Sep. 1947.

\bibitem{nano3}
William Glenn Gray, \emph{Germany's Cold War: The Global Campaign to Isolate East Germany}, 1949-1969, Chapel Hill: University of North Carolina Press, 2003.


\bibitem{nano3}
Janick Marina Schaufelbuehl, Marco Wyss, and Sandra Bott, \emph{Choosing Sides in the Global Cold War: Switzerland, Neutrality, and the Divided States of Korea and Vietnam}. The International History Review, Vol. 37, No. 5, 2015, pp. 1014-1036.


\bibitem{nano3}
Based on figures of Historical Statistics of Switzerland Online, http://www.fsw.uzh.ch/histstat/main.php.

\bibitem{nano3}
Committee for foreign affairs of the national council, minutes of the special meeting of 26 Aug. 1971, 27 Aug. 1971, SFA, E2004B\#1979/31\#1*.


\bibitem{nano3}
Office of the permanent observer of the United Nations to the Division of Political affairs, 1 Dec. 1971, SFA, E2001E-01\#1982/58\#2669.


\bibitem{nano3}
The Swiss ambassador to China to the Division of Political Affairs, 29 May 1973, SFA, E2001E-01\#1987/78\#2400*.


\bibitem{nano3}
Marc Perrenoud, \emph{Switzerland's relationship with Africa during decolonisation and the beginnings of development cooperation}, International Development Policy, Revue internationale de politique de d\'eveloppement, http://poldev.revues.org/140, March 2010
 
\bibitem{nano3}
Purtschert, Patricia, and Harald Fischer-Tiné, \emph{Colonial Switzerland: Rethinking Colonialism from the Margins}, 2015.
 
\bibitem{nano3}
Benjamin Talton, \emph{The Challenge of Decolonization in Africa}, Temple University, http://exhibitions.nypl.org/africanaage/essay-challenge-of-decolonization-africa.html
\end{comment}
\bibitem{nano3}
Young, Allen, \emph{Swiss Neutrality in the Cold War}, New York: Columbia College, 1962

\bibitem{nano3}
\label{reference2}
\emph{Belgium in World War I}, https://en.wikipedia.org/wiki/Belgium\_in\_World\_War\_I

\bibitem{nano3}
\label{reference1}
\emph{Belgium in World War I}, https://en.wikipedia.org/wiki/Belgium\_in\_World\_War\_II

\bibitem{nano3}
\label{reference3}
\emph{Compromessi e compromissioni}, https://www.uek.ch/it/presse/uebrigeartikel/010831corriere.htm, August 2002

\bibitem{nano3}
\label{reference4}
  Palgrave Macmillan, J\"{u}rg Martin Gabriel, and Thomas Fischer, \emph{Swiss Foreign Policy, 1945 -- 2002}, 2003

\bibitem{nano3}
\label{reference5}
Bergier et al., \emph{Switzerland, National Socialism and the Second World War}, 2002

\bibitem{nano3}
\label{reference6}
\emph{Treaty on the Non-Proliferation of Nuclear Weapons (NPT)}, https://www.un.org/disarmament/wmd/nuclear/npt/

\bibitem{nano3}
\label{reference7}
\emph{The Biological Weapons Convention}, http://www.unog.ch/80256EE600585943/(httpPages)/04FBBDD6315AC720C1257180004B1B2F?OpenDocument

\bibitem{nano3}
\label{reference8}
André Schaller, emph{Schweizer Neutralität im West-Ost-Handel: das Hotz-Linder-Agreement vom 23. Juli 1951}, 1987

\bibitem{nano3}
\label{reference9}
Tanner, \emph{Die Schweiz und Rüstungskontrolle: Grenzen und Möglichkeiten eines Kleinstaates}, 1990

\bibitem{nano3}
\label{reference10}
\emph{Hotz-Linder-Agreement (1951)}, https://db.dodis.ch/organization/27882

\bibitem{nano3}
\label{reference11}
\emph{The Nuclear Suppliers Group}, http://www.nuclearsuppliersgroup.org/en/history1

\bibitem{nano3}
\label{reference12}
  Marco Wyss, \emph{Neutrality in the early Cold War: Swiss arms imports and neutrality}. Cold War History, p. 25-49, February 2012

\bibitem{nano3}
\label{reference13}
\emph{SIPRI Database}, http://armstrade.sipri.org/armstrade/page/trade\_register.php

\bibitem{nano3}
\label{reference14}
Harto Hakovirta, \emph{East-west conflict and European neutrality}, 1988

\bibitem{nano3}
\label{reference15}
Alois Riklin, \emph{Neutralität am ende? 500 Jahre Neutralität der Schweiz}

\bibitem{nano3}
\label{reference16}
J\"{u}rg Martin Gabriel, \emph{The American Conception of Neutrality After 1941}, 1988

\bibitem{nano3}
\label{reference17}
Stephen C. Neff, \emph{The rights and duties of neutrals: a general history}, 2000

\bibitem{nano3}
\label{reference18}
Bruno Fritzsche and Christina Lohm, \emph{Cold War and Neutrality: East-West Economic Relations in Europe}, 2006

\bibitem{nano3}
\label{reference19}
Mikael Nilsson and Marco Wyss, \emph{The Armed Neutrality Paradox: Sweden and Switzerland in US Cold War Armaments Policy}, 2015

\bibitem{nano3}
\label{reference20}
Mauro Mantovani, \emph{Schweizerische Sicherheitspolitik im Kalten Krieg}, 1999

\bibitem{nano3}
\label{reference20}
\emph{db.dodis.ch/document/7237}, 1952

\bibitem{nano3}
Raymond R. Probst, \emph{Good Offices in the Light of Swiss International Practice and Experience}, Dordrecht/Boston/Londres, Martinus Nijhoff Publishers, 1989, p. 114

\bibitem{nano3}
  Thomas Fischer, \emph{La Suisse et la représentation des intérêts américains à Cuba dans les années 1961-1977}. Relations internationales 2010/4 (n. 144), p. 73-86. DOI 10.3917/ri.144.0073

\bibitem{nano3}
  Virginie Fracheboud, \emph{La Suisse au service des intérêts américains à Cuba ou le succès de la politique de neutralité et solidarité (1961-1963)}. Relations internationales 2015/3 (n. 163), p. 47-62. DOI 10.3917/ri.163.0047
    
\bibitem{nano3}
Wayne S. Smith, \emph{The Protecting Power and the us Interests Section in Cuba}. Diplomacy Under a Foreign Flag: When Nations Break Relations, \'ed. David D. Newsom, 1990, p. 100

\bibitem{nano3}
\emph{Foreign Relations of the United States}, 1961-1963, vol. X, doc. 7, note 3, Telegram from the Department of State to the Embassy in Cuba, Washington, 3 janvier 1961, 9:05 p.m.

\bibitem{nano3}
Diplomatic Documents of Switzerland (DODIS), \emph{e-Dossier: Swiss Representation of US-Interests in Cuba}

\bibitem{nano3}
Diplomatic Documents of Switzerland (DODIS), \emph{e-Dossier: Switzerland and the Cuban missile Crisis}

\bibitem{nano3}
Diplomatic Documents of Switzerland (DODIS), \emph{Termination of the diplomatic relations between Cuba and the US}

\bibitem{nano3}
Thomas Fisher, \emph{Talking to the bearded Man: The swiss mandate to represent US interests in Cuba}, Working Papers in International History and Politics No. 5, August 2010

\bibitem{nano3}
John Hudson, \emph{The Untold Story of the U.S. and Cuba’s Middleman}, August 3rd 2015

\bibitem{nano3}
Swiss Federal Department of Foreign Affairs, \emph{Dossier for Swiss representations and the media concerning the end of Switzerland’s mandates to represent United States interests in Cuba and Cuban interests in the United States}, July 2015


\bibitem{nano3}
Grégory Theintz, \emph{Le sucre de Nestl\'e: originalité de l'accord sur les nationalisations de biens helvétiques par le régime castriste}, 1960-1967

\bibitem{nano3}
\emph{Swiss Representation of US-Interests in Cuba}, http://dodis.ch/en/e-dossier-swiss-representation-of-us-interests-in-cuba

\bibitem{nano3}
\emph{Foreign Relations of the United States}, 1961-1963, vol. X, doc. 7, note 3, Telegram from the Department of State to the Embassy in Cuba, Washington, 3 janvier 1961, 9:05 p.m.




\bibitem{nano3}
Max Petitpierre, \emph{Conference des Ministres}, Department of Foreign Affairs (then called Federal Political Department), 12 Sep. 1947, dodis.ch/321, S[wiss] D[iplomatic] D[ocuments], http://db.dodis.ch/.

\bibitem{nano3}
\emph{William Glenn Gray}, Germany’s Cold War: The Global Campaign to Isolate East Germany, 1949-1969, Chapel Hill: University of North Carolina Press, 2003.

\bibitem{nano3}
\emph{Janick Marina Schaufelbuehl}, Marco Wyss, and Sandra Bott. “Choosing Sides in the Global Cold War: Switzerland, Neutrality, and the Divided States of Korea and Vietnam”, The International History Review, Vol. 37, No. 5, 2015, pp. 1014-1036.

\bibitem{nano3}
Based on figures of Historical Statistics of Switzerland Online, http://www.fsw.uzh.ch/histstat/main.php.

\bibitem{nano3}
Committee for foreign affairs of the national council, minutes of the special meeting of 26 Aug. 1971, 27 Aug. 1971, SFA, E2004B\#1979/31\#1*.

\bibitem{nano3}
Office of the permanent observer of the United Nations to the Division of Political affairs, 1 Dec. 1971, SFA, E2001E-01\#1982/58\#2669.

\bibitem{nano3}
The Swiss ambassador to China to the Division of Political Affairs, 29 May 1973, SFA, E2001E-01\#1987/78\#2400*.

\bibitem{nano3}
Alfred Glesti, EPD, \emph{Notiz. Diplomatische Beziehungen mit Südkorea. Das Problem der geteilten Staaten}, 5 Oct. 1964, dodis.ch/31039, SDD.

\bibitem{nano3}
Aktennotiz EPD, \emph{Anerkennung der Nord- und Südhälften von Korea und von Vietnam und die Aufnahme diplomatischer Beziehungen}, 29 Jan. 1962, dodis.ch/18910, SDD.

\bibitem{nano3}
\emph{Telegram of the Swiss Embassy in Washington to the Department of Foreign Affairs}, 26March 1973, SFA, E2001E-01\#1987/78\#2400*.




\end{thebibliography}
\end{document}